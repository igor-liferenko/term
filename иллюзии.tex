%&17pt
\pdfpagewidth=297mm
\pdfpageheight=210mm
\input quire \pdfhorigin=1in \pdfvorigin=1in
\shhtotal=\pdfpagewidth \htotal=.5\shhtotal
\vtotal=\pdfpageheight \shvoffset=-\pdfvorigin
\shoutline=0pt
\shstaplewidth=0pt
\shcrop=0pt
\shfootline={}
\quire{23}

\horigin=4.5mm
\vorigin=2.5mm
\hsize=\htotal \advance\hsize by-2\horigin
\vsize=\vtotal \advance\vsize by-8mm

\font\chapfont=omr10 at45pt
\font\handfont=omss17
\font\bukvfont=omr10 at41pt

\nopagenumbers

\raggedbottom

\topglue3cm
{\chapfont 1}
\vskip3cm

\begingroup
\baselineskip=17pt
\newcount\n
\everypar={\advance\n by1 \llap{\number\n.\enspace}\hang}
\parskip=30pt
\handfont

Жил-был Учитель, который пришёл на Землю, родившись в святой земле Индиане, и вырос среди таинственных холмов к востоку от Форт-Уэйна.

Учитель учился этому миру в здешних местах, в обычной школе штата Индиана, а когда вырос --- у своей профессии автомеханика.

Но у Учителя были и другие знания, полученные из других школ, из других стран и из других жизней, которые он прожил. Он помнил их, и эта память делала его мудрым и сильным, так что другие увидели силу его и пришли к нему за советом.

Учитель верил, что у него есть сила помочь себе и всему человечеству, и было ему по вере его, так что и другие увидели его силу и пришли к нему, чтобы он избавил их от бед и исцелил от всяческих болезней.

Учитель верил, что для любого человека полезно думать о себе, что он сын Бога, и как он верил, так оно и было, и в мастерских, где он работал, появлялись толпы тех, кто искал его учения и хотел коснуться его, а улицы были запружены теми, кто жаждал лишь одного: чтобы на них упала его тень, когда он пройдет мимо них и изменит их жизнь.

Но закончилось это тем, что из-за этих толп управляющие мастерскими, где он работал, попросили Учителя оставить работу и идти своей дорогой, ибо так тесно обступала его толпа, что ни у него, ни у других механиков не было места, чтобы чинить автомобили.

И случилось так, что он ушел за пределы города, и люди, следовавшие за ним, стали называть его Мессией и чудотворцем, и как они верили, так оно и было.

Если, когда он говорил, проносилась гроза, то ни единой капли не падало на головы слушающих его. Те, кто стоял далеко от него в этой огромной толпе, слышали его слова так же ясно, как и те, кто был рядом с ним, --- не важно, гремел ли при этом гром или сверкала молния. И всегда он говорил с ними притчами.

И говорил он им, ``в каждом из нас заложена сила нашего согласия на здоровье и болезнь, на богатство и бедность, на свободу и рабство. Это мы управляем той великой силой, и никто другой.''

Тут заговорил мельник и сказал он, ``Легко Тебе говорить, Учитель, ибо Тобой руководят свыше, а нами --- нет, и Тебе не нужно так тяжко трудиться, как трудимся мы. Человеку приходится работать, чтобы жить в этом мире.''

Учитель отвечая им сказал, ``Когда-то жили-были существа в одной деревне на дне большой прозрачной реки.

``Река безмолвно протекала над ними --- над молодыми и старыми, богатыми и бедными, добрыми и злыми.
 Текла своим собственным путём, зная лишь своё собственное хрустальное «Я».

``Все существа, каждый по-своему, крепко цеплялись за стебли и камни на дне реки, ибо цепляться было их образом жизни, а сопротивление течению --- тем, чему каждый учился с рождения.

``Но одно существо сказало наконец, `Мне надоело цепляться. Хотя я не могу этого видеть, но я верю, что поток знает, куда течет. Я оттолкнусь от дна, и пусть он несет меня, куда захочет. Продолжая цепляться, я умру от скуки.'

``Другие существа засмеялись и сказали, `Глупец! Отцепишься --- и поток, которому ты поклоняешься, швырнёт тебя о камни и разобьёт вдребезги, и ты умрешь скорее, чем от скуки!'

``Но он не обратил на их слова внимания и, собравшись с духом, перестал держаться --- и сразу же его закрутило течение и швырнуло о камни.

``Однако он не захотел цепляться снова, и поток высоко поднял его, свободного, со дна, и больше он уже не получал синяков и ушибов.

``А существа на дне, для которых он уже был чужим, кричали, `Смотрите! Он летает, хотя подобен нам! Свершилось чудо! Смотрите, Мессия пришел спасти нас всех!'

``А он несясь в потоке сказал, `Я не более Мессия, чем вы. Река с восторгом принимает нас, свободных, если только мы осмеливаемся довериться ей. Наша истинная работа --- в этом путешествии, в этом приключении.'

``Но они ещё громче кричали, `Спаситель!' --- и всё так же цеплялись за камни и ветки, и когда они снова взглянули вверх, его уже не было, и они остались одни создавать легенды о Спасителе.''

И закончилось тем, что с каждым днем всё больше и больше людей толпились вокруг него, и всё ближе,
всё плотнее, всё настойчивей. И он увидел, что они вынуждают исцелять их, не давая отдыха, и кормить их постоянно чудесами, учиться за них и жить их жизнью. И тогда он ушел на вершину горы и там молился.

И сказал он в сердце своём, Бесконечное Лучезарное Сущее, если будет на то воля твоя, да минует меня чаша сия, да избавишь Ты меня от этой немыслимой задачи. Я не могу жить жизнью ни одной другой души, однако десятки тысяч молят меня о том. Прости меня, что я позволил всему этому случиться. Если будет на то воля Твоя, дай мне вернуться к моим моторам и инструментам и жить, как живут другие.

И голос на холме заговорил с ним, голос который не был ни мужским, ни женским, ни громким, ни тихим, голос который был бесконечно добр. И сказал ему голос, ``Не моя воля, но твоя да свершится. Ибо то, что есть твоя воля, это и моя воля для тебя. Иди своим путём, как и все другие люди, и да будешь ты счастлив на земле.''

И услышав это, Учитель обрадовался и возблагодарил, и сошел вниз с горы, напевая песенку
механика. И когда толпа приступила к нему со своими горестями, умоляя исцелить их, озарить их тут
же своей мудростью и развлечь их чудесами, он улыбнулся народу и сказал любезно: ``Я закончил.''

На мгновение толпа онемела от изумления.

И спросил он у них, ``Если человек сказал Богу, что он больше всего на свете хочет помогать страждущему миру, любой ценой, и Бог ответил, и сказал ему, что он должен делать, следует ли человеку делать так, как ему сказано?''

``Конечно, Учитель!'' закричали они. ``Для него должно быть радостью испытывать даже муки ада, если Бог просит этого!''

``Не важно, каковы эти муки и насколько сложна задача?''

``Честь быть повешенным или быть распятым на дереве и сожжённым, если это то, о чём попросил
 Господь,'' сказали они.

``А что бы вы сделали,'' сказал Учитель, ``если бы Бог заговорил с каждым из вас и прямо сказал вам, `Я ПОВЕЛЕВАЮ, ЧТОБЫ ВЫ БЫЛИ СЧАСТЛИВЫ В МИРЕ, ПОКА ВЫ МОЖЕТЕ.' Что бы вы сделали тогда?''

И молчала толпа, и ни голоса, ни звука не было слышно в холмах и долине, где они стояли.

И сказал тогда Учитель в тишине, ``На тропе нашего счастья найдём мы знание, ради которого мы
избрали эту жизнь. Вот что я узнал в этот день, и я выбираю покинуть вас, и теперь вы пойдёте
своим собственным путём, как вам захочется.''

И он ушёл своей дорогой сквозь толпу и оставил их, и вернулся к повседневному миру людей и машин.

\endgroup
\vfil\eject

\topglue3cm
{\chapfont 2}
\vskip3cm

\begingroup\clubpenalty=10000
\def\par{\endgraf\endgroup}
\advance\parindent by30pt \hang\hangafter=-2
\noindent\vtop to0pt{\kern-8pt\hbox to0pt{\hss\kern\parindent\bukvfont Б\kern3pt}\vss}ыла середина
лета, когда случай свёл меня с Дональдом Шимодой.
За четыре года полётов мне ни разу не встретился ни один пилот, который делал бы то же самое, что
и я: летал вместе с ветром с места на место на стареньком биплане и катал народ --- три доллара
за десять минут полёта. Но однажды, пролетая к северу от Ферриса, штат Иллинойс, я глянул вниз из
открытой кабины своего {\it Флита}, и, поверите ли, он был там, старый {\it Тревл Эйр-4000}, весь
белый-и-золотой на изумрудно-лимонном огороженном поле.

Мне нравится моя свободная жизнь, но иногда бывает и одиноко. Я увидел биплан внизу, несколько
минут подумал и решил, что не будет ничего плохого, если я ненадолго к нему присоединюсь. Убрав
газ до малых оборотов, руль высоты до отказа вниз, мы --- {\it Флит\/} и я --- в широком развороте стали
снижаться. Негромко и ласково, словно ветер в качающихся проводах, затянул мотор своё медленное
«пок-пок», замедляя вращение пропеллера. Я сдвинул защитные очки на лоб, чтобы лучше видеть, куда
садиться. Зелёные джунгли кукурузы со свистом проносились подо мной, мелькнул забор, а затем,
насколько я мог видеть, недавно скошенное поле. Я выровнялся, затем тихо зашуршала земля под
колёсами --- медленнее, медленнее, наконец быстрые выхлопы двигателя, и {\it Флит\/} остановился
рядом с другим самолётом. Я выключил зажигание. Послышалось тихое «клак-клак», пропеллер замер, и наступила тишина июльского дня.

Пилот {\it Тревл Эйра\/} сидел в траве, прислонясь к левому колесу своего самолёта, и наблюдал за мной. С полминуты я тоже молчал, дивясь тайне его спокойствия. Я бы лично не был бы таким невозмутимым и не сидел бы так, наблюдая, как рядом со мной приземляется другой самолёт, всего в десяти ярдах. Я кивнул ему. Он мне понравился, сам не знаю почему.

``Ты выглядишь одиноким,'' сказал я ему издали.

``Ты тоже.''

``Не хочу мешать тебе. Если меня слишком много, полечу дальше.''

``Нет, я ждал тебя.''

Я улыбнулся на это. ``Прости, что опоздал.''

``Ничего.''

Я стянул шлем и очки, вылез из кабины и спрыгнул на землю. После пары-другой часов в кабине моего
{\it Флита\/} это всегда приятно.

``Надеюсь, ты не откажешься от ветчины и сыра,'' сказал он. ``Хлеб с ветчиной и сыром, а может, и с муравьём.''
Никаких рукопожатий, никаких церемоний знакомства.

Он не выглядел слишком крепким. Волосы до плеч, чернее чем резина на колесе, к которому он прислонился. Глаза темные, как у ястреба. Хорошо, когда такие глаза у друга, но у любого другого они заставили бы вас чувствовать себя очень неуютно. Он вполне мог бы быть мастером каратэ, собирающимся продемонстрировать свое жестокое и бесшумное искусство.

Я взял у него сандвич и стаканчик с водой из термоса.
``Всё-таки, ты кто?'' спросил я. ``За годы, что я катаю фермеров на самолёте, я ни разу не видел ни одного такого же бродяги, как я.''

``Да я ни на что особенное и не гожусь,'' сказал он вполне беззаботно. ``Немножко механик, сварщик, чуть халтурил с гусеничными тракторами. Задерживаясь на одном месте больше, чем нужно, я попадаю в переделки. Так что отремонтировал старый самолёт и --- бродяжничаю.''

``А в каких моделях тракторов ты разбираешься?'' Я с детства бредил дизельными тракторами.

``Д-8, Д-9. Да это было совсем недолго, в Огайо.''

``Д-девятые! Такие большие, с дом! С двойным приводом! Это правда, что они могут сдвинуть целую гору?''

``Есть гораздо лучшие способы сдвигать горы,'' сказал он с улыбкой, которая длилась разве что долю секунды.

С минуту я изучал его, прислонясь к нижнему крылу его самолёта. Игра света...\ на него трудно было смотреть вблизи. Как если бы он был окружен серебристым сиянием, создающим слабый светящийся фон.

``Что-нибудь не так?'' спросил он.

``В какие такие переделки ты попадал?''

``О, ничего особенного. Просто сейчас мне нравится всё время менять места. Как и тебе.''

Я взял свой сандвич и обошел самолёт кругом. Это была машина выпуска 1928 или 1929 года, нисколько не
 поцарапанная. Заводы не выпускают таких новеньких самолётов и не ставят их на стоянку в поле. По
 меньшей мере двадцать слоев авиалака сияли как зеркало на полированных вручную деревянных деталях
 машины. Под кабиной золотыми буквами старинным английским шрифтом написано: «Дон», а на регистрационной
 карте значилось: «Д.~У. Шимода». Приборы новёхонькие, как только что из упаковки, настоящие авиационные приборы образца 1928 года. Дубовая полированная приборная доска и ручка управления, регулятор смеси и опережения зажигания слева. Теперь уже нигде не встретишь опережение зажигания, даже в самых лучших реставрированных самолётах. И нигде ни царапины, ни пятнышка на перкалевой обшивке, ни одного подтёка масла на фюзеляже. Ни единой соломинки на полу кабины, как будто его машина никогда не летала. Словно она просто материализовалась на этом самом месте сквозь какую-то дыру во времени протяжённостью полвека. Я почувствовал, как от всего этого у меня по спине забегали мурашки.

``И давно ты уже катаешь фермеров?'' спросил я у него, глядя на самолёт.

``С месяц, может, недель пять.''

Он лгал. Пять недель полётов над полями --- и кто бы вы ни были, на вашем самолёте появятся и
грязь, и солома на полу кабины. Хоть что-нибудь да будет. Но эта машина... Ни капли масла на
ветровом стекле, ни пятнышка от травы на несущих плоскостях, ни расплющенных на пропеллере
насекомых. Летом в Иллинойсе это невозможно ни для какого самолёта. Я изучал {\it Тревл Эйр\/} ещё
минут пять, а потом вернулся и сел на траву под крылом лицом к пилоту. Мне не было страшно. Мне
всё ещё нравился этот парень, но только что-то здесь было явно не так.

``Почему ты говоришь мне не правду?''

``Я сказал тебе правду, Ричард,'' сказал он. Моё имя тоже нарисовано на моем самолёте.

``Слушай, человек не может катать пассажиров на своём самолёте целый месяц без того, чтобы не замаслить
 его или хотя бы, друг мой, не запачкать хоть что-нибудь. Хоть одно пятнышко на обшивке, а? Господи, ну
 хоть травинка на полу?''

Он спокойно улыбнулся мне. ``Есть вещи, о которых ты ещё не знаешь.''

В этот момент он был чужаком, пришельцем с другой планеты. Я поверил тому, о чем он говорил, но у меня не было никакого объяснения для пребывания этой ювелирной вещицы посреди лётного поля.

``Это правда. Но когда-нибудь я их узнаю. Всё узнаю. И тогда ты сможешь забрать себе мой самолёт,
 Дональд, потому что чтобы летать он мне уже не понадобится.''

Он посмотрел на меня с интересом, его чёрные брови поднялись.
``Да ну? Расскажи мне.''

Я был в восторге. Кто-то хочет услышать мои теории!

``Долгое время люди не умели летать. Думаю, это происходило просто оттого, что они считали это
невозможным, поэтому, конечно, и не научились первому закону аэродинамики. Мне хочется верить,
что где-то существует другой закон: вам не нужны самолёты, чтобы летать...\ или добираться до
других планет. Мы можем научиться делать всё это и без машин. Если захотим.''

Он улыбнулся едва заметно и серьёзно кивнул.
``И ты думаешь, что научишься всему этому, если будешь катать пассажиров над полями по три доллара за
 рейс.''

``Единственное знание, которое имеет для меня смысл, это то, которое получаешь самостоятельно,
делая то, что хочешь. Нет и не может быть на земле ни одной души, которая могла бы научить меня
большему, чем мой самолёт и небо. И если бы такой человек был, я бы тут же направился разыскивать
его. Или её.''

Тёмные глаза спокойно смотрели на меня.
``А тебе не приходило в голову, что тебя на самом деле кто-то ведёт, раз ты хочешь всему этому
 научиться?''

``Да, меня ведут. А кого --- нет? Я всегда чувствовал, что за мной наблюдают или что-то в этом роде.''

``И ты думаешь, тебя приведут к учителю, который тебе поможет.''

``Если не случится так, что этим учителем окажусь я сам, да.''

``Может так и случится,'' сказал он.

\vskip28pt
\hbox{\kern7cm \font\karta=karta15 scaled3000 \karta ^^4a}
\vskip15pt

Современный новый пикап, подняв облако пыли, неслышно подъехал к нам по дороге и остановился у поля. Дверца отворилась, и оттуда вышли старик и девочка лет десяти. Было так безветренно, что пыль осталась висеть в воздухе.

``Это вы тут катаете?'' спросил старик.

Поле нашёл Дональд Шимода, и я промолчал.

``Да, сэр,'' сказал он весело. ``Сегодня неплохо полетать, правда? Хотите?''

``А если бы захотел, вы не станете выкидывать номера, делать всякие там петли-кувырки?'' глаза
 старика озорно блестели, пока он разглядывал нас, чтобы увидеть, принимаем ли мы его за
 прос\-та\-ка-де\-ре\-вен\-щи\-ну.

``Захотите --- будем, не захотите --- не будем.''

``Так вы наверное дерёте за это чёртову уйму денег.''

``Три доллара наличными, сэр, за десятиминутный рейс. Это значит тридцать три и треть цента в
 минуту. Многие, кто рискнул, говорят, что дело того стоит.''

У меня было странное чувство стороннего наблюдателя, который болтается тут и слушает, как другой
нахваливает свой товар. Мне понравилось, что он сказал всё так спокойно, не повышая голоса. Я
настолько привык к своей собственной рекламе (``{\sl Парни, гарантирую вам наверху температуру на десять
градусов ниже! Отправимся туда, где летают только птицы и ангелы! И всё это за три доллара,
 мизерная часть содержимого вашего кошелька или кармана...}''), что и забыл о возможности делать это как-то иначе.

Это напряженная работа, летать и катать пассажиров в одиночку. Я привык к этому, но тут было и
 другое: если у меня не будет пассажиров, мне нечего будет есть. Сейчас, когда мой обед достался мне даром, я мог позволить себе побездельничать, слегка расслабился и просто наблюдал.

Девочка стояла в стороне и тоже наблюдала. Белокурая, с карими глазами, с серьёзным лицом, она была здесь только из-за дедушки. Она не хотела летать.

Чаще бывает как раз наоборот: жаждущие летать дети и осторожные взрослые; когда вы зарабатываете себе на жизнь, у вас вырабатывается чутье на такие дела, --- и я точно знал, что девочка не полетит с нами, сиди мы тут хоть целое лето.

``Так кто из вас, джентльмены...?'' спросил старик.

Шимода налил себе в кружку воды.
``Ричард полетит с вами. У меня ещё обед. Разве что вы готовы подождать.''

``Нет, сэр, я готов прямо сейчас. Мы можем пролететь над моей фермой?''

``Конечно,'' сказал я. ``Только покажите, куда лететь, сэр.''
Я выгрузил из передней кабины {\it Флита\/} спальный мешок, ящик с инструментами и котелки, помог старику забраться на пассажирское кресло и пристегнул его ремнями. Затем я скользнул на заднее сиденье и застегнул собственный ремень.

``Крутни пропеллер, Дон, хорошо?''

``Давай.'' Держа кружку с водой, он встал у пропеллера. ``Как надо?''

``Зажигание, и хватит. Поверни легонько. От толчка он сразу вырвется из руки.''

Всегда, когда кто-нибудь прокручивал пропеллер {\it Флита}, он делал это слишком быстро, и по тем или иным непонятным причинам мотор не заводился. Но этот человек провернул его так плавно, словно только этим и занимался всю жизнь. Щелкнула пружинка стартера, искра проскочила в цилиндры, и старый мотор заработал совсем легко. Дональд вернулся к своему аэроплану, сел и заговорил с девочкой.

В рёве мотора и вихре летящей соломы {\it Флит\/} вскарабкался в воздух на сто футов (если сейчас мотор заглохнет, мы сядем в кукурузу), пять сотен футов (теперь мы можем повернуть и сесть на выгоне --- к западу тут выгон для коров), восемьсот футов. Я выравниваюсь и следую туда, куда указывает палец старика, навстречу юго-восточному ветру.

Три минуты полёта, и мы делаем круг над фермерским хозяйством --- сараями цвета раскаленного угля, над домом, словно выточенным из слоновой кости, и морем мяты. Позади дома огород --- сахарная кукуруза, салат и помидоры.

Когда мы кружили над фермой, старик на переднем сиденье смотрел вниз, в пространство между
крыльями и расчалками {\it Флита}. На крылечке под нами появилась женщина в белом переднике поверх
синего платья и помахала нам. Старик помахал ей в ответ. Позднее они будут говорить, как им хорошо
было видно друг друга, несмотря на высоту. Наконец он обернулся ко мне и покивал, показывая, что
довольно, спасибо, мы можем возвращаться.

Я сделал широкий круг над Феррисом, чтобы оповестить жителей о наших воздушных прогулках,
и стал снижаться по спирали, показывая, где это происходит. Когда я пошёл на посадку, делая крутой
вираж над полем, {\it Тревл Эйр\/} взмыл с земли и сразу же повернул в сторону фермы, над которой мы только что летали.

Когда-то я летал в составе пятерки, и на мгновение у меня возникло то же ощущение...\ один самолёт взлетает с пассажирами, другой приземляется. Мы коснулись земли с небольшим сотрясением и откатились до дальнего края поля, ближе к дороге. Мотор заглох. Старик отстегнул ремень, и я помог ему выбраться. Он достал из комбинезона бумажник и, качая головой, отсчитал долларовые бумажки.

``Вот это прогулка, сынок.''

``А как же иначе. Мы предлагаем товар высшего сорта.''

``Это всё твой друг. Вот уж кто умеет предложить свой товар!''

``Почему он?''

``Вот что я тебе скажу. Твой друг продал бы золу самому дьяволу. Держу пари, разве не так?''

``С чего вы это взяли?''

``Из-за девочки, почему же ещё. Чтобы моя внучка Сара полетела на самолёте!'' Он не спускал
 глаз с Трэвл Эйра, кружащего над фермой в го\-лу\-бо\-ва\-то-се\-реб\-рис\-той дымке. Он говорил как человек, обнаруживший у себя в саду засохшую березу, которая вдруг расцвела и покрылась румяными спелыми яблоками.

``С самого рождения эта девочка до смерти боялась высоты. Вопит. Просто в ужас приходит. Она бы
 скорее голой рукой схватила шершня, чем влезла на дерево. Не поднялась бы по лестнице на чердак,
 даже если бы началось наводнение. Девочка чудо как хорошо управляется с машинами, да и с
 животными неплохо ладит, но высота --- она ее боится как чумы! И вот посмотрите-ка --- она летит.''

Он говорил о нынешних и добрых старых временах; припомнил как много лет назад такие же вот бродяги-авиаторы, бывало, появлялись у Гейлсбурга и Монмаута на таких же, как у нас, самолётиках --- у них тоже было по два крыла, --- только те выделывали с ними черт знает что.

Я смотрел, как далекий {\it Тревл Эйр\/} становится всё больше, как он снижается над полем по
более крутой спирали, чем это сделал бы я, будь у меня на борту девочка, которая боится высоты...
Как он скользнул над кукурузным полем, оградой, коснулся скошенной травы и приземлился на три
точки так, что дух захватило. Дональд Шимода, должно быть, уже немало налетал, если уж он мог так
посадить {\it Тревл Эйр}.

Самолет подкатил и остановился возле нас, для этого не пришлось делать никаких лишних усилий --- пропеллер тихонько звякнул и остановился. И ни одной мертвой мошки на восьмифутовой лопасти.

Я вскочил, чтобы помочь, отстегнул у девочки ремень, открыл для нее дверцу передней кабины и
 показал, куда поставить ногу, чтобы не повредить обшивку крыла.

``Как тебе понравилось?'' сказал я.

Она даже не слышала, что я ей сказал.

``Деда, я не боюсь! Мне было не страшно, честно! Дом был как игрушечный, и мамочка помахала мне, а
 Дон сказал, что я боялась просто потому, что однажды упала с высоты и умерла, и что больше я не
 должна бояться! Я буду летчиком, деда! У меня будет самолёт, и я сама буду чинить его, и везде
 летать, и катать людей на своём самолёте! Можно?''

Шимода улыбнулся старику и пожал плечами.

``Это он сказал тебе, что ты будешь летчиком, да, Сара?''

``Нет, я сама. Ведь я уже хорошо разбираюсь в моторах, ты же знаешь!''

``Ладно, поговоришь об этом со своей матерью. Нам пора домой.''

Эти двое поблагодарили нас, и один пошел, а другая вприпрыжку побежала к своему грузовичку, оба изменившиеся от того, что случилось с ними на поле и в небе.

Появились ещё два автомобиля, потом ещё один, и весь день у нас было полно людей, которые хотели увидеть Феррис с высоты птичьего полёта.

Мы сделали двенадцать или тринадцать рейсов, при этом торопились, как только могли, потом я
помчался в город за горючим для {\it Флита}. Потом было ещё несколько пассажиров, и ещё, и настал вечер, а мы без остановки взлетали и садились до самого заката.

Где-то на дорожном указателе я прочитал: население 220, и к тому времени, как стемнело, я стал
думать, что мы прокатили их всех, может, ещё нескольких и из пригорода. В спешке я забыл спросить Дона о Саре и о том, что он ей сказал, придумал ли какую-то историю --- или то, что он думал о смерти, было правдой. И краем глаза, пока пассажиры рассаживались, я внимательно наблюдал за его самолётом.

Нигде ни пятнышка --- он явно умудрялся в полёте уклоняться от насекомых, которых после
 часа-другого лёта мне приходилось счищать с ветрового стекла.

Небо уже почти потемнело, когда мы закончили. К моменту, когда я набил сухими кукурузными стеблями
 свою жестяную печку, положил сверху угольные брикеты и разжег огонь, стало совсем темно, и огонь
 бросал яркие блики на самолёты, стоящие рядом, и на золотые стебли соломы вокруг нас.

Я заглянул в ящик с продуктами.
``Суп, тушёнка или спагетти,'' сказал я. ``Может груши или персики. Хочешь горячих персиков?''

``Всё равно,'' сказал он мягко. ``Что-нибудь или ничего.''

``Парень, ты что не голоден? Это был трудный денёк!''

``Ты не предлагаешь ничего такого, ради чего стоило проголодаться, разве только это будет хорошая
 тушёнка.''

Я открыл тушёнку своим швейцарским офицерским ножом, проделал то же самое с банкой спагетти
 и поставил обе банки на огонь.

Карманы мои были набиты деньгами. Для меня это был один из наиболее приятных моментов дня. Я вытащил бумажки и пересчитал их, нимало не заботясь о том, чтобы расправить их или сложить. Вышло 147 долларов, и я стал считать в уме, что дается мне с немалым трудом.

``Постой...\ постой...\ дай-ка подумать, четыре и два в уме...\ сорок девять рейсов сегодня.
Я побил тот рекорд, когда у меня был стодолларовый день, Дон, только я да {\it Флит\/}!
А ты должен был побить и двухсотдолларовый, ведь ты в основном берешь двух за раз?''

``В основном,'' сказал он.

``Кстати, о том учителе, которого ты ищешь,'' не\-много погодя начал Дон.

``Не ищу я никакого учителя,'' сказал я. ``Я деньги считаю! Я могу неделю прожить на это. Целую
 неделю я могу ни фига не делать! Пусть хоть дождь зарядит, мне не страшно!''

Он улыбнулся, глядя на меня.
``Когда ты наконец накупаешься в деньгах, не будешь ли ты так добр передать мне тушёнку?''

\bye
